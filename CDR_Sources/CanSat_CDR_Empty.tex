\documentclass[11pt]{article}
\usepackage{ROSPINCanSat, lipsum, xcolor}
\cansatstyle

\title{Guidelines for the Critical Design Report
}
\author{Team: ROSPIN}
\date{January 22, 2024}

\begin{document}

\cansattitle


%%%%%%%%%% INTRO PAGE - DELETE BEFORE SUBMISSION %%%%%%%%%%%

The process of building a satellite is very complex and costly. That is why, in a real satellite mission, before, during, and after the satellite is built, these documents provide detailed information about the satellite being developed and ensure that it complies with all the requirements regarding the mission and the launch environment.

The process of designing and building a CanSat is much simpler than the one followed for a satellite. Nevertheless, we believe that exposing students to good engineering procedures will be very beneficial for their educational experience.

These guidelines provide information about the expected content of each \textbf{Critical Design Report (CDR)} chapter. 
This information will ensure that your work is aligned with your mission goals and can help us identify possible problems at an early stage. 
It will also allow us to determine that your CanSat will be able to fly according to the mechanical and safety requirements.

Attached to this document, there is a sample design document with a given structure that you can use to describe all the aspects of your CanSat project. 
Your report should be no longer than \textbf{30 pages}, not including appendices and references. Appendices should be used for detailed information 
to keep the document's main body as concise as possible. This detailed information may be, e.g., details of scientific background, technical drawings, or 
component datasheets. The documentation should be written clearly and concisely, allowing a person who does not know the experiment to understand its purpose and design.

The design document should provide the JURY with all the relevant information regarding the experiment. 
During all experiment phases, the design document is the only documentation for describing the experiment in detail. 
Additional sections can be added by the team if appropriate. However, the present sections should be included in your CDR for reference. The design document will be the main evaluating criteria for the Romanian CanSat and Rocketry Competition jury.

\vspace{3cm}
{\Large{\textbf{Note:} Do not include this page in your Report!}}
%%%%%%%%%% END OF INTRO PAGE - DELETE BEFORE SUBMISSION %%%%%%%%%%%


\newpage

\tableofcontents
\pagestyle{plain}

\newpage





% First section
\section{Progress report}
% Write the progress report here

\subsection{New progress statement for the team profile}
% Here comes the content regardin the new progress statement for the team profile

\subsection{Tasks list}
% Here comes the content regardin the tasks list

\subsection{Detailed project status}
% Here comes the content regardin the detailed project status




% 2nd section
\section{Introduction}
% Write the introduction text here

\subsection{Purpose of the mission}
% Here comes the content regardin the purpose of the mission.

\subsection{Team organisation and roles}
% Here comes the team organisation and roles description content.

\subsection{Mission objectives}
% Here comes the text regarding the mission objectives.




% 3rd section
\section{CanSat description / Payload description}
% Write the CanSat description text here for the CanSat challenge
% or the Payload escription text here for the Rocketry challenge

\subsection{Mission overview}
% Here comes the text regarding the mission overview.

\subsection{Mechanical / structural design}
% Here comes the text regarding the mechanical and structural design.

\subsection{Electrical design}
% Here comes the text regarding the electrical design.

\subsection{Software design}
% Here comes the text regarding the software design.

\subsection{Recovery system}
% Here comes the text regarding the recovery system.

\subsection{Ground support equipment}
% Here comes the text regarding the ground support equipment.

% section only for the Rocketry Challenge
% \section{Rocket Description}

% \subsection{Rocket Mission overview}
% % Here comes the text regarding the mission overview.

% \subsection{Rocket's mechanical / structural design}
% % Here comes the text regarding the mechanical and structural design.

% \subsection{Rocket's recovery system}
% % Here comes the text regarding the recovery system.


% 4th section
\section{Project planning}
% Write about the project planning here

\subsection{Time schedule of the project preparation}
% Here comes the text regarding the time schedule.

\subsection{Resource estimation}
% Here comes the text regarding the resource estimation.

\subsubsection{Budget}
% Here comes the text regarding the budget.

\subsubsection{External support}
% Here comes the text regarding the external support.

\subsection{Test plan}
% Here comes the text regarding the test plan.

\subsection{Time management}
% Here comes the text regarding the time management.




% 5th section
\section{Data analysis and outreach}
% Here comes data analysis and outreach content.

\subsection{Data Analysis Plan}
% Here comes the text regarding the data analisys plan..

\subsection{Outreach Program}
% Here comes the text regarding the outreach program.




% 6th section
\section{Conclusion}
% Here comes the conclusion content.

\subsection{Summary of the CDR}
% Here comes the summary for the CDR

\subsection{Recommendations for next steps}
% Here comnes the recomandations for the next phase/steps

\end{document}
