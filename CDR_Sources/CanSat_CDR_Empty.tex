\documentclass[11pt]{article}
\usepackage{ROSPINCanSat, lipsum, xcolor}
\cansatstyle

\title{Guidelines for the Critical Design Report
}
\author{Team: ROSPIN}
\date{January 22, 2024}

\begin{document}

\cansattitle

\newpage

\tableofcontents
\pagestyle{plain}

\newpage





% First section
\section{Progress report}
% Write the progress report here

\subsection{New progress statement for the team profile}
% Here comes the content regardin the new progress statement for the team profile

\subsection{Tasks list}
% Here comes the content regardin the tasks list

\subsection{Detailed project status}
% Here comes the content regardin the detailed project status




% 2nd section
\section{Introduction}
% Write the introduction text here

\subsection{Purpose of the mission}
Overall goal is to help businesses pick a strategic placing for their entities, and to revolutionize GPS services with faster and more precise data. 

\subsection{Team organisation and roles}

\begin{itemize}
\item \textbf{Team Name:} {ShuttleSat - shuttle is spaceship + sat is satellite, and we brought them together.}
\item \textbf{Team Composition:}
\begin{itemize}
\item[-] {Rares: Previous Experience → Hardware\&Embedded Google course \textbar\hspace{0cm} Background → pure sciences and maths \textbar\hspace{0cm} Interests → physics and mechanical engineering \textbar\hspace{0cm} Field of work within team → pick satellite's components, build the satellites, keeps track of available components, take care of components integrity, research and plan on physical components \textbar\hspace{0cm} Expected hours dedicated → 20-30.}
\item[-] {David: Background → electronics, signals and devices \textbar\hspace{0cm} Interests → electronic engineering \textbar\hspace{0cm} Field of work within team → R\&D, expertise in mission's area, data collect/analysis, timeline of data results, find and compare technical solutions, aerospace and electrical expertise \textbar\hspace{0cm} Expected hours dedicated → 20-30.}
\item[-] {Alex: Previous Experience → Project managing Innovation Labs mentorship \textbar\hspace{0cm} Background → analytical thinking and logical reasoning \textbar\hspace{0cm} Interests → team management and software engineering \textbar\hspace{0cm} Field of work within team → pick technologies, design and develop the software, meetings, initial ideas/information to start from maintain a developing plan, keep track of project directions, design/develop communication system \textbar\hspace{0cm} Expected hours dedicated → 20-30.}
\newline
\end{itemize}

\item \textbf{Team's Activity:} {Half meetings are online and half are physical, based on the expected length and progress review or the intended planification volume of the meet.}
\end{itemize}

\subsection{Mission objectives}
\begin{enumerate}
    \item\textbf{Traffic measurement and mapping, for known congested areas, either entities identified as cars or humans.} {Secondary mission objective is to achieve an infrastructure capable of monitoring human and cars traffic. The infrastructure should be based on a single satellite that covers a single area at a time. We are looking forward to collect data such as: the number of entities inside the area and the flux of entities relative to time.}
    \item\textbf{Proving that the concept can be applied at a larger scale, such as a network of orbital satellites doing traffic measurements, is the target objective to call the launch successful.} {Main target is to successful capture stable and high quality pictures of the designated area. Smaller targets are to transmit this data to ground controller, and develop an innovative algorithm and AI in order to obtain 2 main variables: number of entities and their flux, within an area.}
    \item\textbf{We are expecting to find: majority of challenges, data to simulate actual implementation, costs and impact, infrastructure improvements.} {We are looking further to understand and list the majority of challenges that are faced when designing and building an orbiting satellite for current mission objective. Obtain the necessary data to run simulations of the cost to build and maintain a satellite network that measures and maps traffic, as well as it's impact and benefits. A flawless data transfer infrastructure is expected to be built trough the overcame challenges. Nonetheless we expect to improve our current skills and develop new ones, necessary for satellites development and engineering.}
    \item\textbf{Measure the values of CH4, CO2 and N2 and the quantity of entities inside a designated area.} {We are seeking to identify the \% of CH4, CO2 and N2 from the air with a precision of up to 98\% and the quantity of entities, in a targeted area with a precision of around 50\%. Also we are looking to test the transmission of this data, its capabilities and limitations, to the ground controller, continuously, until under 100m of altitude. But the most important test will be the satellite as a whole, that every system work as expected and what improvements can be done for a successful infrastructure.}
\end{enumerate}




% 3rd section
\section{CanSat description / Payload description}
% Write the CanSat description text here for the CanSat challenge
% or the Payload escription text here for the Rocketry challenge

\subsection{Mission overview}
% Here comes the text regarding the mission overview.

\subsection{Mechanical / structural design}
% Here comes the text regarding the mechanical and structural design.
\begin{enumerate}
\item \textbf{Mechanical Design:} The CanSat is made of carbon fiber/ alluminum alloy because of their mechanical properties such as low weight, high durability and high tolerances in thermal expansion. The structure was designed to maintain a certain position while mid-air and to withstand the stress of launch and landing. It comes with a removable top and bottom to facilitate access inside. The components are attached to the structure with screws, nuts and mounts to assure stability during the mission.
\vspace{0.25cm}
\item \textbf{Components:} Components used for building the CanSat include: a main board, sensors, a video camera, a battery and a GPS module. The main board contains the microcontroller which collects the data from the sensors and sends it to the ground station via it’s communication module. The sensors include: a temperature sensor, a CH4 sensor, a N2 sensor, a CO2 sensor and an altimeter which measures both pressure and altitude. The GPS module registers the current position of the CanSat. The video camera takes frames of the surface below to be processed afterwards at the ground station. The battery provides power to the CanSat during flight. The list of components stands as it follows:

\begin{itemize}
\item \textbf{Microcontroller:} Raspberry PI Zero
\item \textbf{Temperature sensor:} sensor DS18B20
\item \textbf{CH4 sensor:} sensor MQ-4
\item \textbf{N2 sensor:} sensor MiCS-5524
\item \textbf{CO2 sensor:} sensor MQ-9
\item \textbf{Altimeter:} sensor BMP-388
\item \textbf{GPS Module:} GPS GY-NEO6MV2 module
\item \textbf{Communication module:} already integrated in main board
\item \textbf{Gyroscope:} sensor MPU6050
\item \textbf{Video Camera:} OV5647 video camera
\item \textbf{Rechargable battery:} portable power bank
\end{itemize}

\item \textbf{Placement:} The main board is attached to the top of the CanSat structure with the sensors located adjacent to it. The battery is attached to a side of the CanSat while the camera is attached to the opposite side of it. The structure was designed to maintain balance during flight, to minimize the weight and to provide space for the components and for further human intervention.
\vspace{0.25cm}
\item \textbf{Drawings:} This is a mechanical drawing of the CanSat structure in which we have emphasized on the placement of the major components.

\begin{figure}[hbt!]
\includegraphics[width=8cm]{Component_placement}
\centering
\end{figure}

\item \textbf{Explanation:} The main board contains the microcontroller which collects the data from the sensors and sends it to the ground station via it’s communication module. The temperature sensor measures the temperature of the environment, the CH4 sensor, the N2 sensor and the CO2 sensor measures the atmospheric gasses in the air and the altimeter measures both pressure and altitude at the CanSat’s point. The GPS module registers the current position of the CanSat. The video camera takes frames of the surface below to be processed afterwards at the ground station. The battery provides power to the CanSat during flight.

\end{enumerate}

\subsection{Electrical design}
% Here comes the text regarding the electrical design.
\begin{enumerate}
\item \textbf{Electrical Interface:}
\item \textbf{RF Link:}
\item \textbf{Power Budget:}
\item \textbf{Power Consumption and Duration:}
\item \textbf{Battery:} 

\begin{figure}[hbt!]
\includegraphics[width=15cm]{Schema_electrica}
\centering
\end{figure}
\end{enumerate}

\subsection{Software design}
% Here comes the text regarding the software design.

\subsection{Recovery system}
% Here comes the text regarding the recovery system.
\begin{enumerate}
\item \textbf{Description:} The Recovery System of the CanSat is designed in order to be easy to make and to ensure a safe and controlled descent back to the ground station. It consists of a flat parachute with a drag coefficent of 0.77 and an area of 0.06 $m^2$, made with rip-stop fabric and attached to the CanSat via nylon cords. The parachute is deployed as soon as the CanSat is released from the rocket.
\vspace{0.25cm}
\item \textbf{Method of attachment:} The parachute is attached to the CanSat structure using a harness that is made of nylon fibres. The harness is connected to the structure itself with screws and PVC collars providing a stable attachment for the recovery system
\vspace{0.25cm}
\item \textbf{Picture:} Here is an indicative picture of how the parachute will look during the descent

\begin{figure}[hbt!]
\includegraphics[width=5cm]{Parachute_example}
\centering
\end{figure}

\item \textbf{Expected flight time:} The CanSat's estimated flight time is around 10 minutes in which it reaches its maximum altitude and returns back to the ground station. After a short simulation, taking in consideration the characteristics of the parachute and of the Payload, the time for the descent is 100.7 seconds in perfect weather conditions and it reaches its maximum velocity of 10 m/s in 10 seconds. Here are the altitude and velocity variation graphs:

\begin{figure}[hbt!]
\includegraphics[width=15cm]{Height_variation}
\centering
\end{figure}

\begin{figure}[hbt!]
\includegraphics[width=15cm]{Velocity_variation}
\centering
\end{figure}

\end{enumerate}

\subsection{Ground support equipment}
% Here comes the text regarding the ground support equipment.

% section only for the Rocketry Challenge
% \section{Rocket Description}

% \subsection{Rocket Mission overview}
% % Here comes the text regarding the mission overview.

% \subsection{Rocket's mechanical / structural design}
% % Here comes the text regarding the mechanical and structural design.

% \subsection{Rocket's recovery system}
% % Here comes the text regarding the recovery system.


% 4th section
\section{Project planning}
% Write about the project planning here

\subsection{Time schedule of the project preparation}
% Here comes the text regarding the time schedule.

\subsection{Resource estimation}
% Here comes the text regarding the resource estimation.

\subsubsection{Budget}
% Here comes the text regarding the budget.
\begin{center}
	\begin{tabular}{|c|c|}
		\hline
		  Component & Cost(RON) \\
		\hline
		  Microcontroller & 100\\
		  Temperature Sensor & 12\\
		  CH4 Sensor & 11\\
		  N2 Sensor & 80\\
		  CO2 Sensor & 15\\
		  Altimeter & 70\\
		  GPS Module & 50\\
		  Gyroscope & 17\\
		  Video Camera & 80\\
		  Rechargable Battery & 100\\
		  Structure & 200\\
		  Additional materials & 250\\
		  Personnel & 200\\
		  Launch site fees & 200\\
		\hline
		  Total Budget & 1385\\
		\hline
           \end{tabular}
\end{center}

\subsubsection{External support}
% Here comes the text regarding the external support.
Our CanSat project, ShuttleSat, receives support just from the following departments:
\begin{itemize}
\item \textbf{University Politehnica Bucharest:} 
\begin{itemize}
\item[-] Professors from the Electronics and Telecomunications department are providing useful informations for the electric diagrams and components configurations
\item[-] Professors from the Computer Science department are offering advices in the choice of components and development of software
\end{itemize}
\end{itemize}

At the moment, the team lacks financial support forthe purchase of components and also lacks support in the area of aerodynamics testing

\subsection{Test plan}
% Here comes the text regarding the test plan.

\subsection{Time management}
% Here comes the text regarding the time management.




% 5th section
\section{Data analysis and outreach}
% Here comes data analysis and outreach content.

\subsection{Data Analysis Plan}
% Here comes the text regarding the data analisys plan..

\subsection{Outreach Program}
% Here comes the text regarding the outreach program.




% 6th section
\section{Conclusion}
% Here comes the conclusion content.

\subsection{Summary of the CDR}
% Here comes the summary for the CDR

\subsection{Recommendations for next steps}
% Here comnes the recomandations for the next phase/steps

\end{document}
